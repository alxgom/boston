\documentclass[a4paper,spanish]{exam}
\usepackage[spanish]{babel}
%\usepackage[utf8]{inputenc}
\usepackage{multicol}
%\usepackage[latin1]{inputenc}
\usepackage{fontspec}%la posta para las tildes con lualatex
\usepackage[margin=0.5in]{geometry}
\usepackage{amsmath,amssymb}
\usepackage{multicol}
\usepackage{natbib}
\usepackage{graphicx}
\usepackage{hyperref}
\usepackage{epstopdf}
\usepackage{capt-of}
\usepackage[space]{grffile}
\usepackage[usenames]{color}
%los de aca abajo capaz no los uso
\newcommand{\class}{Matemática: Evaluación de Funciones Racionales {\tiny (Recuperatorio)}}
\newcommand{\term}{2° Trimestre 2015}
\newcommand{\examnum}{Tema 2}
\newcommand{\examprof}{Alexis Gomel}
\newcommand{\examdate}{15/9/2015}
\newcommand{\timelimit}{60 Minutes}%no lo uso
\newcommand{\Ts}{\rule{0pt}{2.8ex}}       % Top strut
\newcommand{\Bs}{\rule[-1.5ex]{0pt}{0pt}} % Bottom strut
%el header de las hojas.
\pagestyle{head}
\firstpageheader{}{}{}
\runningheader{\class}{\examnum\ - pagina \thepage\ de \numpages}{\examdate}
\runningheadrule


\begin{document}
	\noindent
	\begin{minipage}{0.92\linewidth}
		\begin{tabular*}{\textwidth}{l @{\extracolsep{\fill}} r @{\extracolsep{6pt}} l}
			\textbf{\class} & \textbf{Profesor: \examprof}\\
			\textbf{\examnum}  & \textbf{}   \\
			%& Teaching Assistant & \makebox[2in]{\hrulefill}
			\textbf{Nombre: } \makebox[2in]{\hrulefill} & \textbf{\examdate} 
		\end{tabular*}\\
	\end{minipage}
	\begin{minipage}[r]{0.08\linewidth}
		\begin{flushright}
			\includegraphics[width=\linewidth]{bost.png}
		\end{flushright}
	\end{minipage}\\
	\rule[2ex]{\textwidth}{2pt}
	
	%%%%%%%%%%%%%%%%%%%%%%%%%%%%%%%%%%%%%%%%%%%
	
	\begin{center}
		\textsl{\textbf{\underline{Justificar}}} cada respuesta. El trabajo practico se entrega \textbf{\underline{escrito en tinta}}.\\
		Si se traban con un ejercicio sigan con el siguiente.
		\textbf{Preguntas:} $\bigcirc \bigcirc \bigcirc  \bigcirc $
	\end{center}
	
	\begin{table}[h]
		\centering
		%\caption{My caption}
		\label{my-label}
		\begin{tabular}{|l|c|c|c|c|}
			\hline
			Ejercicio        & 1 & 2 & 3 & Nota \\ \hline
			Puntaje máximo   & 4 & 2 & 4 &   10   \\ \hline
			Puntaje obtenido &   &   &   &      \\ \hline
		\end{tabular}
	\end{table}
	
	Si se traban con algún ejercicio, pasen al siguiente y vuelvan a intentar mas tarde con el que dejaron.
	
	\begin{enumerate}
		
		%uno de inversa
		
		
		%homografica
		\item Graficar la función homografica \[ y=\frac{-3}{x+1}+4 \] especificando el Dominio, la Imagen, las raíces y las asintotas. \\
		
		Encontrar para que valores de $x$ la función es menor o igual a $5$.
		
		%una a graficar de las fdificiles
		
		\item Cual función corresponde al gráfico. 
		%aca va una tabla y unos graficos
		
		\begin{minipage}{0.5\textwidth}
			\centering
			%\begin{table}[!h]
			%\caption{mc1}
			\label{mc1}
			\begin{tabular}{|c|c|c|}
				\hline
				$\frac{-x^5}{ x^2-16}$  & $\frac{-x^4}{ x^2-16}$ & $\frac{-x^2}{ x^2-16}$ \Ts \Bs   \\ \hline
				&   &      \\ \hline
			\end{tabular}\\
			%\end{table}
			%\begin{figure}[h]
			\centering
			\includegraphics[width= 0.95\linewidth]{problematemarec21.png}
			%\end{figure}
			%graficos
		\end{minipage}
		\begin{minipage}{.5\textwidth}
			\centering
			%\begin{table}[!h]
			%\caption{mc1}
			%\label{mc1}
			\begin{tabular}{|c|c|c|}
				\hline
				$\frac{5}{(x-5)(x-2)}$  & $\frac{5x}{(x-5)(x-2)}$ & $\frac{5x^2}{(x)(x-2)}$ \Ts \Bs   \\ \hline
				&   &      \\ \hline
			\end{tabular}\\
			%\end{table}
			%\begin{figure}[h]
			\centering
			\includegraphics[width= 0.95\linewidth]{problematemarec22.png}
			%\end{figure}
\end{minipage}
		
		
		\item  Graficar la función \[ y=\frac{2x^4}{x^3-27}  \]. Indicando el Dominio, la Imagen, las raíces y las asintotas.
		
		\item (bonus)\textbf{Extra:}
		Si ya terminaste los demás, este ejercicio sirve como un bonus para darte un empujón si estas cerca de aprobar, o para redondear la nota para arriba.\\
		
		Demostrar que la multiplicación por una función racional y la potenciación (para potencias pertenecientes a los enteros) de una función racional $f(x)=\frac{P(x)}{Q(x)}$ da como resultado una función racional. ¿Porque necesito restringirme a los enteros?.
		
		Es decir que $h(x)=f(x).g(x)$ también es una función racional (si $f$ y $g$ son funciones racionales), sin importar los grados de los polinomios de $P,Q,L$ y $M$; y $(f(x))^{n}$ también lo es. 
\end{enumerate}
\end{document}