\documentclass[a4paper,12pt]{exam}
\usepackage[spanish]{babel}
\usepackage{multicol}
\usepackage{fontspec}%la posta para las tildes con lualatex
\usepackage[margin=0.5in]{geometry}
\usepackage{amsmath,amssymb}
\usepackage{natbib}
\usepackage{graphicx}
\usepackage{color}
\usepackage{gensymb}
\usepackage{tikz}%para graficos
\usetikzlibrary{decorations.markings}
\usetikzlibrary{shapes.geometric}
\usepackage{tkz-euclide}
\usetkzobj{all}

%los de aca abajo capaz no los uso
\newcommand{\class}{Evaluación de Matemática. Previa de 4to.}
\newcommand{\term}{Diciembre 2015}
\newcommand{\examnum}{Tema 1}
\newcommand{\examprof}{Alexis Gomel}
\newcommand{\examdate}{9/12/2015}
\newcommand{\timelimit}{60 Minutes}%no lo uso


%el header de las hojas.
\pagestyle{head}
\firstpageheader{}{}{}
\runningheader{\class}{\examnum\ - Page \thepage\ of \numpages}{\examdate}
\runningheadrule

\begin{document}
\noindent
\begin{tabular*}{\textwidth}{l @{\extracolsep{\fill}} r @{\extracolsep{6pt}} l}
\textbf{\class} & \textbf{Profesor: \examprof}\\
\textbf{\examnum} & \textbf{\examdate} \\
%\textbf{Time Limit: \timelimit} & Teaching Assistant & \makebox[2in]{\hrulefill}
\textbf{Nombre: } \makebox[2in]{\hrulefill}
\end{tabular*}\\
\rule[2ex]{\textwidth}{2pt}

%%%%%%%%%%%%%%%%%%%%%%%%%%%%%%%%%%%%%%%%%%%

\textbf{Justificar} cada respuesta en \textbf{tinta}. 

\begin{table}[h]
\centering
%\caption{My caption}
\label{my-label}
\begin{tabular}{|l|c|c|c|c|c|c|}
\hline
Ejercicio        & 1 & 2 & 3 & 4 & 5 &  Nota \\ \hline
Puntaje máximo   & 2,5 & 2 & 2 & 1 & 2,5 &   10   \\ \hline
Puntaje obtenido &   &   &   &   &   &    \\ \hline
\end{tabular}
\end{table}

Si se traban con algún ejercicio, pasen al siguiente y vuelvan a intentar mas tarde con el que dejaron.

\begin{enumerate}
\item \textbf{Resolver el siguiente triangulo}
%$\hat{\alpha}=60\degree $
%$\hat{\gamma}=100\degree $\\


\begin{tikzpicture}[thick]
\coordinate (O) at (0,0);
\coordinate (A) at (3,0);
\coordinate (B) at (2.5,1.5);
\draw (O)--(A)--(B)--cycle;

\tkzLabelSegment[below=2pt](O,A){$10m$}
\tkzLabelSegment[left=2pt](O,B){$15m$}
\tkzLabelSegment[above right=2pt](A,B){}

\tkzMarkAngle[fill=orange,size=0.3cm,opacity=.4](A,O,B)
\tkzLabelAngle[pos = -0.7](A,O,B){$\hat{\gamma}=30\degree $}

\tkzMarkAngle[fill= orange,size=0.3cm,%
opacity=.4](B,A,O)
\tkzLabelAngle[ pos= -0.4 ](B,A,O){$\hat{\alpha}$}

\tkzMarkAngle[fill= orange,size=0.3cm,%
opacity=.4](O,B,A)
\tkzLabelAngle[pos = -0.4](O,B,A){$\hat{\beta}$}
\end{tikzpicture}

Encontrar el lado restante y los ángulos internos.

\item \begin{multicols}{2}
\begin{enumerate}
\item $2^x+2^{x+1}+\frac{5}{4}2^{x+2}=256$
\item $log_3(x^2)+log_3(x)-6=0$

\columnbreak

\item $e^{3.ln(4)-2.ln(3)+10^{28}.ln(1)}$
\item $log_b(log_a(a^{(b^k)}))$

\end{enumerate}
\end{multicols}

\item 
\begin{enumerate}

\item \textbf{Calcular}
$\dfrac{(-1+i).(2-4i)}{-2+3i}$
\item \textbf{Hallar z:}
$i^{3}(2+4z)=z.i^{10}-i+3$
\end{enumerate}


\item \textbf{Racionalizar, indicando el resultado en su mínima expresión}
\begin{enumerate}
\item $ \dfrac{\sqrt{3}-1}{1+\sqrt{3}} $

\item $ \dfrac{\sqrt{15}}{3.\sqrt{5}+5.\sqrt{3}}  $
\end{enumerate}

\item Sea la función $y=-2x^2-2x+4$.

Encontrar el máximo o mínimo según corresponda, las raíces y el punto del vértice. 
Escribir la función en su forma factorizada y canónica.
 
\item (bonus)\textbf{Extra:}
Si ya terminaste los demás, este ejercicio sirve como un bonus para darte un empujón si estas cerca de aprobar, o para redondear la nota para arriba.\\
 

Deducir porque en el caso del triangulo rectángulo siempre resulta que: $\cos(\alpha)=\sin(\beta)$ y $\sin(\alpha)=\cos(\beta)$ .

 
\end{enumerate}

\rule[2ex]{\textwidth}{2pt}

“Knowing a great deal is not the same as being smart; intelligence is not information alone but also judgement, the manner in which information is coordinated and used.”
― Carl Sagan, Cosmos 




%\section*{Respuestas}
%1: 1)3 ;2) 25 ;3) -6 ;4)$ 34 (9+25)$ ;5)$ 1+1,46 $ ;6)$ 1/1,46=0,685$ ;7)$2*1,46$ ;8)1

%2: ver grafico
%$\frac{(6)+(4i)}{(1)+(1+i)}=3-i$
%3:$log_2(x+2)$

%4: 1)2  ;2)1 ;3)2 ;4)$\frac{m^2.n^4}{100}$
\end{document}
