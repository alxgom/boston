%****************************************************************
%Primer examen para diciembre/febrero. 
%4to año. Colegio Boston
%****************************************************************

\documentclass[a4paper,11pt,spanish,sans]{exam}
\usepackage[spanish]{babel}
%\usepackage[utf8]{inputenc}
\usepackage{multicol}
%\usepackage[latin1]{inputenc}
\usepackage{fontspec}%la posta para las tildes con lualatex
\usepackage[margin=0.5in]{geometry}
\usepackage{amsmath,amssymb}
\usepackage{multicol}
\usepackage{natbib}
\usepackage{graphicx}
\usepackage{hyperref}
\usepackage{epstopdf}
\usepackage{capt-of}
\usepackage{gensymb}
\usepackage{float}
\usepackage{wrapfig}
\usepackage{pst-fractal}
%\usepackage{animate}
\usepackage[usenames]{color}
%para graficos
\usepackage{pgf,tikz}
\usepackage{mathrsfs}
\usetikzlibrary{arrows}
\usepackage{pst-fractal}
\usetikzlibrary{decorations.markings}
\usetikzlibrary{shapes.geometric}
\usetikzlibrary{shapes,snakes}
\usepackage{tkz-euclide}
\usetkzobj{all}

\newcommand{\class}{Examen Diciembre 4to }
\newcommand{\term}{12/2015}
\newcommand{\examnumuno}{Tema 1}
\newcommand{\examnumdos}{Tema 2}
\newcommand{\examnumvulcano}{Tema 3}
\newcommand{\examprof}{Alexis Gomel}
\newcommand{\examdate}{20/11/2015}
\newcommand{\timelimit}{60 Minutes}%no lo uso
\newcommand{\Ts}{\rule{0pt}{2.8ex}}       % Top strut
\newcommand{\Bs}{\rule[-1.5ex]{0pt}{0pt}} % Bottom strut
%el header de las hojas.
\pagestyle{head}
\firstpageheader{}{}{}
%\runningheader{\class}{\examnumuno\ - pagina \thepage\ de \numpages}{\examdate}
%\runningheadrule

\begin{document}

%*********************************************************
%Tema 1
%*********************************************************
\noindent 
\begin{minipage}{0.92\linewidth}
	\begin{tabular*}{\textwidth}{l @{\extracolsep{\fill}} r @{\extracolsep{6pt}} l}
		\textbf{\class} & \textbf{Profesor: \examprof}\\
		\textbf{\examnumuno}  & \textbf{}   \\
		%& Teaching Assistant & %VII la venganza de adrian  \makebox[2in]{\hrulefill}
		\textbf{Nombre: } \makebox[2in]{\hrulefill} & \textbf{Fecha: } \makebox[2in]{\hrulefill}  
	\end{tabular*}\\
\end{minipage}
\begin{minipage}[r]{0.08\linewidth}
	\begin{flushright}
		\includegraphics[width=\linewidth]{bost.png}
	\end{flushright}
\end{minipage}\\
\rule[2ex]{\textwidth}{2pt}

\begin{center}
	\textsl{\textbf{\underline{Justificar}}} cada respuesta. La evaluación se entrega \textbf{\underline{escrita en tinta}}.\\
	Si se traban con un ejercicio sigan con el siguiente.
	May the force be with you.
\end{center}
\begin{table}[h]
	\centering
	%\caption{My caption}
	\label{tema1}
	\begin{tabular}{|l|c|c|c|c|c|c|c|}
		\hline
		Ejercicio        & 1 & 2 & 3 & 4 & Nota & Hojas \\ \hline
		Puntaje máximo   & 1 & 3 & 3 & 3 & 10 &  Entregadas \\ \hline
		Puntaje obtenido &   &   &   &   &    &    \\ \hline
	\end{tabular}
\end{table}

\section{Numeros Reales y conjuntos (1 puntos) }
Graficar los conjuntos $\mathbb{N}$(Naturales), $\mathbb{R}$(Reales), $\mathbb{Q}$(Racionales), $\mathbb{C}$(Complejos), $\mathbb{I}$(Irracionales), $\mathbb{Z}$(Enteros), $\mathbb{I}m$ (imaginarios) como diagramas de Venn.

\section{Radicacion (3 puntos)}
	
	\begin{enumerate}
		\begin{multicols}{2}
			\item $\dfrac{\sqrt{a-1}-\sqrt{a+1}}{\sqrt{a-1}+\sqrt{a+1}}$
			%\item $\dfrac{\sqrt{3}-1}{1+\sqrt{3}} $
			\item $\dfrac{6}{2\sqrt{3}+3\sqrt{2}} $
			%\item $\dfrac{21}{3\sqrt{7}+4\sqrt{3}} $
			%\item $\dfrac{15}{3\sqrt{5}+5\sqrt{3}} $
			\columnbreak
			%	\item $\dfrac{\sqrt{5}-2}{2+\sqrt{5}}$
			%	\item $\dfrac{10}{2\sqrt{5}+5\sqrt{2}}$
			%\item $\sqrt{\dfrac{\sqrt{2}+1}{\sqrt{2}-1}}$
			\item $\sqrt{\dfrac{\sqrt{8}+2}{\sqrt{8}-2}}$
		\end{multicols}	
	\end{enumerate}

\section{Cuadraticas (3 puntos)}
	\begin{enumerate}
		\item Bicuadratica: $-2x^4-2x^2+4=0$
		\item Graficar la siguiente función y expresarla en sus tres formas (normal, canónica y factorizada).
		
		$y=-2x^2+2x+3 $
		%$y=-2(x-1/2)^2+7/2$
	%	 $4/3(x + 1)(x - 3)$
	\end{enumerate}

\section{Logaritmos y exponenciales (3 puntos)}

	\begin{enumerate}
		
		%\item $3.\log_2(x)-2.\log_4(x^2)=2$
		\item $[\log_{-2}(-8)-\log_3(\frac{1}{3})]^{\frac{1}{2}}$
		%\item$ \dfrac{\log_5(1)+\log_{\frac{1}{81}}3}{\log_3(81)-\log_{\frac{1}{9}}(81)3}$
		%\item $4^x-7:2^x-8$
		\item $e^{2.\ln(3)-2\ln(5)+10^{15}ln(1)}$
	    %\item $e^{1/2.\ln(4)-2\ln(5)+10^{19}ln(1)}$
		%\item Encontrar el valor de x para que $2^x+2^{x+1}+\dfrac{5}{4}2^{x+2}=256$
		\item $\log_9(x^2)-\log_3(x^4)=1$
		%\item $\log_25(x^2)-\log_5(x^4)=1$
		\item $(3^{x+3})\frac{1}{27}.3^{2x}=9 $
		%\item $(5^{x+3}\frac{1}{125})5^2x=25 $
	\end{enumerate}

\section{Bonus}
	Demostrar que 
	$\dfrac{\sqrt{ab}}{a\sqrt{b}+b\sqrt{a}}=\dfrac{\sqrt{a}-\sqrt{b}}{a-b}$
	
	
	\textbf{\underline{Hoja de formulas}:} . \\
	
	Cuadráticas:
	
	$y=ax^2+bx+c$
	
	$y=a(x-x_v)^2+y_v$
	
	$y=a(x-x_1)(x-x_2)$
	
	$x_v=\frac{-b}{2a}$
	
	Cambio de base: $\log_a(b)=\dfrac{\log_c(b)}{\log_c(a)}$

\rule[2ex]{\textwidth}{1pt}

“Science is what we have learned about how to keep from fooling ourselves”   -Richard Feynman 

%********************************************************
%Tema 2
%*******************************************************

\setcounter{section}{0}
\newpage
\noindent 
\begin{minipage}{0.92\linewidth}
	\begin{tabular*}{\textwidth}{l @{\extracolsep{\fill}} r @{\extracolsep{6pt}} l}
		\textbf{\class} & \textbf{Profesor: \examprof}\\
		\textbf{\examnumdos}  & \textbf{}   \\
		%& Teaching Assistant & %VII la venganza de adrian  \makebox[2in]{\hrulefill}
		\textbf{Nombre: } \makebox[2in]{\hrulefill} & \textbf{Fecha: } \makebox[2in]{\hrulefill} 
	\end{tabular*}\\
\end{minipage}
\begin{minipage}[r]{0.08\linewidth}
	\begin{flushright}
		\includegraphics[width=\linewidth]{bost.png}
	\end{flushright}
\end{minipage}\\
\rule[2ex]{\textwidth}{2pt}

\begin{center}
	\textsl{\textbf{\underline{Justificar}}} cada respuesta. La evaluación se entrega \textbf{\underline{escrita en tinta}}.\\
	Si se traban con un ejercicio sigan con el siguiente.
	May the force be with you.
\end{center}
\begin{table}[h]
	\centering
	%\caption{My caption}
	\label{tema2}
	\begin{tabular}{|l|c|c|c|c|c|c|c|}
		\hline
		Ejercicio        & 1 & 2 & 3 & 4 & Nota & Hojas \\ \hline
		Puntaje máximo   & 1 & 3 & 3 & 3 & 10 &  Entregadas \\ \hline
		Puntaje obtenido &   &   &   &   &    &    \\ \hline
	\end{tabular}
\end{table}
	
	
		
\section{Numeros Reales y conjuntos (1 puntos) }
	Graficar los conjuntos $\mathbb{N}$(Naturales), $\mathbb{R}$(Reales), $\mathbb{Q}$(Racionales), $\mathbb{C}$(Complejos), $\mathbb{I}$(Irracionales), $\mathbb{Z}$(Enteros), $\mathbb{I}m$ (imaginarios) como diagramas de Venn.
		
\section{Radicacion (3 puntos)}	
	\begin{enumerate}
		\begin{multicols}{2}
		\item $\sqrt{\dfrac{\sqrt{2}+1}{\sqrt{2}-1}}$
			
			\columnbreak
		%\item $\dfrac{\sqrt{5}-2}{2+\sqrt{5}}$
		%	\item $\dfrac{10}{2\sqrt{5}+5\sqrt{2}}$
		%\item $\dfrac{\sqrt{a-1}-\sqrt{a+1}}{\sqrt{a-1}+\sqrt{a+1}}$
		\item $\dfrac{\sqrt{3}-1}{1+\sqrt{3}} $
		%\item $\dfrac{21}{3\sqrt{7}+4\sqrt{3}} $
		\item $\dfrac{15}{3\sqrt{5}+5\sqrt{3}} $
			
		\end{multicols}
		
	\end{enumerate}
	
\section{Cuadraticas (3 puntos)}
	\begin{enumerate}
		\item Bicuadratica: $-2x^4-2x^2+4=0$
		\item Graficar la siguiente función y expresarla en sus tres formas (normal, canónica y factorizada).
		
		%$y=-2x^2+2x+3 $
		$y=-2(x-1/2)^2+7/2$
		%	 $4/3(x + 1)(x - 3)$
	\end{enumerate}
		
\section{Logaritmos y exponenciales (3 puntos)}
	\begin{enumerate}
		
		%\item $3.\log_2(x)-2.\log_4(x^2)=2$
		%\item $[\log_{-2}(-8)-\log_3(\frac{1}{3})]^{\frac{1}{2}}$
		\item$ \dfrac{\log_5(1)+\log_{\frac{1}{81}}(3)}{\log_3(81)-\log_{\frac{1}{9}}(81)}$
		%\item $4^x-7:2^x-8$
		%\item $e^{2.\ln(3)-2\ln(5)+10^{15}ln(1)}$
		\item $e^{1/2.\ln(4)-2\ln(5)+10^{19}ln(1)}$
		%\item Encontrar el valor de x para que $2^x+2^{x+1}+\dfrac{5}{4}2^{x+2}=256$
		%\item $\log_9(x^2)-\log_3(x^4)=1$
		\item $\log_{25}(x^2)-\log_5(x^4)=1$
		%\item $(3^{x+3}\frac{1}{27})3^2x=9 $
		\item $(5^{x+3})\frac{1}{125}.5^{2x}=25 $
	\end{enumerate}
	
\section{Bonus}
	Demostrar que 
	$\dfrac{\sqrt{ab}}{a\sqrt{b}+b\sqrt{a}}=\dfrac{\sqrt{a}-\sqrt{b}}{a-b}$
	
		
\textbf{\underline{Hoja de formulas}:} . \\
		
Cuadráticas:
		
	$y=ax^2+bx+c \qquad $$\qquad y=a(x-x_v)^2+y_v \qquad$		$\qquad y=a(x-x_1)(x-x_2)$
	
	$x_v=\frac{-b}{2a}$
	
	Cambio de base: $\log_a(b)=\dfrac{\log_c(b)}{\log_c(a)}$
	
	\rule[2ex]{\textwidth}{1pt}
	
	“Science is what we have learned about how to keep from fooling ourselves”   -Richard Feynman 
%***************************************************************
%tema 3(para florencia vulcano)
%***************************************************************
\setcounter{section}{0}
\newpage
\noindent 
\begin{minipage}{0.92\linewidth}
	\begin{tabular*}{\textwidth}{l @{\extracolsep{\fill}} r @{\extracolsep{6pt}} l}
		\textbf{\class} & \textbf{Profesor: \examprof}\\
		\textbf{\examnumdos}  & \textbf{}   \\
		%& Teaching Assistant & %VII la venganza de adrian  \makebox[2in]{\hrulefill}
		\textbf{Nombre: } \makebox[2in]{\hrulefill} & \textbf{Fecha: } \makebox[2in]{\hrulefill} 
	\end{tabular*}\\
\end{minipage}
\begin{minipage}[r]{0.08\linewidth}
	\begin{flushright}
		\includegraphics[width=\linewidth]{bost.png}
	\end{flushright}
\end{minipage}\\
\rule[2ex]{\textwidth}{2pt}

\begin{center}
	\textsl{\textbf{\underline{Justificar}}} cada respuesta. La evaluación se entrega \textbf{\underline{escrita en tinta}}.\\
	Si se traban con un ejercicio sigan con el siguiente.
	May the force be with you.
\end{center}
\begin{table}[h]
	\centering
	%\caption{My caption}
	\label{tema3}
	\begin{tabular}{|l|c|c|c|c|c|c|c|}
		\hline
		Ejercicio        & 1 & 2 & 3 & 4 & Nota & Hojas \\ \hline
		Puntaje máximo   & 1 & 3 & 3 & 3 & 10 &  Entregadas \\ \hline
		Puntaje obtenido &   &   &   &   &    &    \\ \hline
	\end{tabular}
\end{table}



\section{Numeros Reales y conjuntos (1 puntos) }
	Graficar los conjuntos $\mathbb{N}$(Naturales), $\mathbb{R}$(Reales), $\mathbb{Q}$(Racionales), $\mathbb{C}$(Complejos), $\mathbb{I}$(Irracionales), $\mathbb{Z}$(Enteros), $\mathbb{I}m$ (imaginarios) como diagramas de Venn.
	
\section{Radicacion (3 puntos)}
	
	\begin{enumerate}
		\begin{multicols}{2}
			\item $\sqrt{\dfrac{\sqrt{2}+1}{\sqrt{2}-1}}$			
			\columnbreak
			%\item $\dfrac{\sqrt{5}-2}{2+\sqrt{5}}$
			%	\item $\dfrac{10}{2\sqrt{5}+5\sqrt{2}}$
			%\item $\dfrac{\sqrt{a-1}-\sqrt{a+1}}{\sqrt{a-1}+\sqrt{a+1}}$
			\item $\dfrac{\sqrt{3}-1}{1+\sqrt{3}} $
			%\item $\dfrac{21}{3\sqrt{7}+4\sqrt{3}} $
			\item $\dfrac{15}{3\sqrt{5}+5\sqrt{3}} $
		
		\end{multicols}
		
	\end{enumerate}
	
\section{Cuadraticas (3 puntos)}
	\begin{enumerate}
		\item Bicuadratica: $-2x^4-2x^2+4=0$
		\item Graficar la siguiente función y expresarla en sus tres formas (normal, canónica y factorizada).
		
		%$y=-2x^2+2x+3 $
		$y=-2(x-1/2)^2+7/2$
		%	 $4/3(x + 1)(x - 3)$
	\end{enumerate}
	
\section{Logaritmos y exponenciales (3 puntos)}
	\begin{enumerate}
		
		%	\item $3.\log_2(x)-2.\log_4(x^2)=2$
		%\item $[\log_{-2}(-8)-\log_3(\frac{1}{3})]^{\frac{1}{2}}$
		\item$ \dfrac{\log_5(1)+\log_{\frac{1}{81}}(3)}{\log_3(81)-\log_{\frac{1}{9}}(81)}$
		%\item $4^x-7:2^x-8$
		%\item $e^{2.\ln(3)-2\ln(5)+10^{15}ln(1)}$
		\item $e^{1/2.\ln(4)-2\ln(5)+10^{19}ln(1)}$
		%\item Encontrar el valor de x para que $2^x+2^{x+1}+\dfrac{5}{4}2^{x+2}=256$
		%\item $\log_9(x^2)-\log_3(x^4)=1$
		\item $\log_{25}(x^2)-\log_5(x^4)=1$
		%\item $(3^{x+3}\frac{1}{27})3^2x=9 $
		\item $(5^{x+3})\frac{1}{125}.5^{2x}=25 $
	\end{enumerate}
	
\section{Bonus}
	Demostrar que 
	$\dfrac{\sqrt{ab}}{a\sqrt{b}+b\sqrt{a}}=\dfrac{\sqrt{a}-\sqrt{b}}{a-b}$
	
	
	\textbf{\underline{Hoja de formulas}:} . \\
	
	Cuadráticas:
	
	$y=ax^2+bx+c \qquad $$\qquad y=a(x-x_v)^2+y_v \qquad$		$\qquad y=a(x-x_1)(x-x_2)$
	
	$x_v=\frac{-b}{2a}$
	
	Cambio de base: $\log_a(b)=\dfrac{\log_c(b)}{\log_c(a)}$
	
	\rule[2ex]{\textwidth}{1pt}
	
	“Science is what we have learned about how to keep from fooling ourselves”   -Richard Feynman 

\end{document}
