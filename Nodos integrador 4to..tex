\documentclass[a4paper,10pt,spanish,sans]{exam}
\usepackage[spanish]{babel}
%\usepackage[utf8]{inputenc}
\usepackage{multicol}
%\usepackage[latin1]{inputenc}
\usepackage{fontspec}%la posta para las tildes con lualatex
\usepackage[margin=0.5in]{geometry}
\usepackage{amsmath,amssymb}
\usepackage{multicol}
\usepackage{natbib}
\usepackage{graphicx}
\usepackage{hyperref}
\usepackage{epstopdf}
\usepackage{capt-of}
\usepackage[usenames]{color}
%los de aca abajo capaz no los uso
\newcommand{\class}{Matemática: Guía de Números Complejos}
\newcommand{\term}{3° Trimestre 2015}
\newcommand{\examnum}{}
\newcommand{\examprof}{Alexis Gomel}
\newcommand{\examdate}{15/7/2015}
\newcommand{\timelimit}{60 Minutes}%no lo uso
\newcommand{\webpdf}{https://drive.google.com/file/d/0B2MOYme4kZd-Q0VHSTJRb0hINTA/view?usp=sharing}%no lo uso
\newcommand{\Ts}{\rule{0pt}{2.6ex}}       % Top strut
\newcommand{\Bs}{\rule[-1.2ex]{0pt}{0pt}} % Bottom strut

\begin{document}
	\begin{center}
		\section*{Nodos para la Integradora de Matematica de 4to}
	\end{center}
	
	\begin{itemize}
		\subsection*{Unidad 1: Numeros Reales}
		\item Radicacion. Propiedades y casos.
		\item Potencia: Propiedades y casos.
		\item Racionalización.
		\item Inecuaciones con modulo.
		\item Polinomios: Operaciones y factorizacion.
		\item Funciones: Dominio, Imagen, Raíces. 
		
		\subsection*{Unidad 2: Funcion Cuadratica}
		\item Representaciones: Canónica, factorizada y polinomica
		\item Ecuaciones: Formula Resolvente
		\item Ecuaciones bicuadráticas
		\item Gráficos
		\item Intersección entre recta y parábola
		
		\subsection*{Unidad 3: Funcion exponencial y Logaritmica}
		\item Propiedades, definición de la función exponencial y logaritmica.
		\item Gráficos
		\item Logaritmos en varias bases
		\item Cambio de Base
		\item Ecuaciones con exponenciales y logaritmos.
		
		\subsection*{Unidad 4: Funciones Racionales}
		\item Función inversa y homografica
		\item Función Racional, definición.
		\item Gráficos de funciones Racionales.
		\item Inecuaciones de funciones racionales.
		
		\subsection*{Unidad 6: Funciones Trigonometricas}
		\item Triángulos rectángulos: Teorema de Pitagoras y razones trigonométricas.
		\item Teorema del Seno y Coseno.
		\item Unidades: Sistema sexagesimal y circular.
		\item Definición de seno y coseno en la circunferencia unidad.
		
		\subsection*{Unidad 5: Numeros Complejos}
		\item Definición y Representaciones.
		\item Gráfico en el plano complejo.
		\item Operaciones entre números complejos: suma, resta, multiplicacion, división, conjugación.
		\item Ecuaciones con números complejos.
		
		
	\end{itemize}
	
	
\end{document}