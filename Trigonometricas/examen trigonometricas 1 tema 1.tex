\documentclass[a4paper,11pt,spanish,sans]{exam}
\usepackage[spanish]{babel}
%\usepackage[utf8]{inputenc}
\usepackage{multicol}
%\usepackage[latin1]{inputenc}
\usepackage{fontspec}%la posta para las tildes con lualatex
\usepackage[margin=0.5in]{geometry}
\usepackage{amsmath,amssymb}
\usepackage{multicol}
\usepackage{natbib}
\usepackage{graphicx}
\usepackage{hyperref}
\usepackage{epstopdf}
\usepackage{capt-of}
\usepackage{gensymb}
\usepackage{float}
\usepackage{wrapfig}
\usepackage{pst-fractal}
%\usepackage{animate}
\usepackage[usenames]{color}
%para graficos
\usepackage{pgf,tikz}
\usepackage{mathrsfs}
\usetikzlibrary{arrows}
\usepackage{pst-fractal}
\usetikzlibrary{decorations.markings}
\usetikzlibrary{shapes.geometric}
\usetikzlibrary{shapes,snakes}
\usepackage{tkz-euclide}
\usetkzobj{all}

\newcommand{\class}{Matemática: Evaluación de Trigonometria {\tiny (Ep. IV: A new hope)}}
\newcommand{\term}{3° Trimestre 2015}
\newcommand{\examnumuno}{Tema 4}
\newcommand{\examnumdos}{Tema 3}
\newcommand{\examnumvulcano}{Tema 2}
\newcommand{\examprof}{Alexis Gomel}
\newcommand{\examdate}{14/10/2015}
\newcommand{\timelimit}{60 Minutes}%no lo uso
\newcommand{\Ts}{\rule{0pt}{2.8ex}}       % Top strut
\newcommand{\Bs}{\rule[-1.5ex]{0pt}{0pt}} % Bottom strut
%el header de las hojas.
\pagestyle{head}
\firstpageheader{}{}{}
%\runningheader{\class}{\examnumuno\ - pagina \thepage\ de \numpages}{\examdate}
%\runningheadrule

\begin{document}
	\noindent
	\begin{minipage}{0.92\linewidth}
		\begin{tabular*}{\textwidth}{l @{\extracolsep{\fill}} r @{\extracolsep{6pt}} l}
			\textbf{\class} & \textbf{Profesor: \examprof}\\
			\textbf{\examnumuno}  & \textbf{}   \\
			%& Teaching Assistant & \makebox[2in]{\hrulefill}
			\textbf{Nombre: } \makebox[2in]{\hrulefill} & \textbf{\examdate} 
		\end{tabular*}\\
	\end{minipage}
	\begin{minipage}[r]{0.08\linewidth}
		\begin{flushright}
			\includegraphics[width=\linewidth]{bost.png}
		\end{flushright}
	\end{minipage}\\
	\rule[2ex]{\textwidth}{2pt}

\begin{center}
	\textsl{\textbf{\underline{Justificar}}} cada respuesta. La evaluacion se entrega \textbf{\underline{escrita en tinta}}.\\
	Si se traban con un ejercicio sigan con el siguiente.
	\textbf{Preguntas:} $\bigcirc \bigcirc \bigcirc  \bigcirc $
\end{center}
\begin{table}[h]
	\centering
	%\caption{My caption}
	\label{my-label}
	\begin{tabular}{|l|c|c|c|}
		\hline
		Ejercicio        & 1 & 2 & Nota \\ \hline
		Puntaje máximo   & 4 & 6 &  10  \\ \hline
		Puntaje obtenido &   &   &      \\ \hline
	\end{tabular}
\end{table}

\begin{enumerate}
	
	\item Resolver los siguientes triángulos rectángulos. (Calcular los lados, los ángulos y sus razones trigonométricas). \label{rectangulos}\\
	
	\begin{minipage}{0.45\linewidth}
		
		\begin{tikzpicture}[thick]
		\coordinate (O) at (0,0);
		\coordinate (A) at (3.5,0);
		\coordinate (B) at (0,2.6);
		\draw (O)--(A)--(B)--cycle;
		
		\tkzLabelSegment[below=2pt](O,A){$b$}
		\tkzLabelSegment[left=2pt](O,B){$a$}
		\tkzLabelSegment[above right=2pt](A,B){$h$}
		
		\tkzMarkRightAngle[fill=orange,size=0.6,opacity=.4](A,O,B)% square angle here
		\tkzLabelAngle[pos = 0.35](A,O,B){$\hat{\gamma}$}
		
		\tkzMarkAngle[fill= orange,size=0.8cm,%
		opacity=.4](B,A,O)
		\tkzLabelAngle[pos = 0.6](B,A,O){$\hat{\alpha}$}
		
		\tkzMarkAngle[fill= orange,size=0.8cm,%
		opacity=.4](O,B,A)
		\tkzLabelAngle[pos = 0.5](O,B,A){$\hat{\beta}$}
		
		\end{tikzpicture}
		
	\end{minipage}
	\begin{minipage}{0.55\linewidth}
		\begin{enumerate}
			%\item $a=3km$,  $\quad b=4km$ .	Expresar los resultados de los ángulos en el sistema sexagesimal.
			%\item $a=2cm$, $\quad b=1cm$
			\item $a=30km$,  $\quad b=20km$ .	Expresar los resultados de los ángulos en el sistema sexagesimal.
			%\item $a=5cm$, $\quad \sin(\beta)=\frac{1}{\sqrt{2}}$. Expresar los resultados de los ángulos en Radianes.
			%\item $a=5cm$, $\quad \cos(\alpha)=\frac{\sqrt{2}}{2}$
			\item $a=5cm$, $\quad \cos(\alpha)=\frac{\sqrt{3}}{2}$ 
		\end{enumerate}
	\end{minipage}
	
	\item Resolver los siguientes triángulos. El esquema del triangulo es solo para que sepan como los puntos $A,B$ y $C$ se corresponden con los angulos. \label{obtusos}

	
	\begin{minipage}{\linewidth}
		\begin{enumerate}
			%\item $\overline{AB}=8cm$,  $\overline{AC}=12cm$ y $\overline{BC}=5cm$ . Encontrar el valor de los ángulos internos y dibujar aproximadamente el triangulo.
			\item Calcular la altura de un edificio sabiendo que a una cierta distancia la punta del edificio se encuentra a un angulo de $65\degree$ respecto del suelo como se muestra la figura, y unos $50m$ ($\overline{AB}$) mas lejos, la misma se observa a una inclinación $30\degree$ desde el suelo.
			\item $\hat{a}=30\degree$,  $\overline{ac}=15cm$ y $\overline{ab}=10cm$ . Dibujar aproximadamente el triangulo y encontrar el valor de $x= \overline{bc}$.
			%\item $\hat{c}=40\degree$,  $\overline{ac}=30cm$ y $\overline{ab}=20cm$ . Dibujar aproximadamente el triangulo y encontrar el valor de $x= \hat{b}$.
			%\item $\hat{a}=60\degree$, $\hat{b}=70\degree$, y $\overline{ac}=20m$. Dibujar aproximadamente el triangulo y resolverlo (hallar todos los valores restantes).
			%\item $\overline{ab}=10cm$,  $\overline{ac}=7cm$ y $\overline{bc}=8cm$ . Dibujar aproximadamente el triangulo y encontrar el valor de $x= \hat{c}$.
		\end{enumerate}
	\end{minipage}
	
	\begin{minipage}{0.5\linewidth}
		
		\begin{tikzpicture}[thick]
		\coordinate (O) at (0,0);
		\coordinate (A) at (3.5,0);
		\coordinate (B) at (-0.6,2.3);
		\draw (O)--(A)--(B)--cycle;
		
		\node at (O) [below left=2pt]{$A$};
		\node at (A) [right=2pt]{$B$};
		\node at (B) [above left=2pt]{$C$};
		
		\tkzLabelSegment[below=2pt](O,A){}
		\tkzLabelSegment[left=2pt](O,B){}
		\tkzLabelSegment[above right=2pt](A,B){}
		
		\tkzMarkAngle[fill=orange,size=0.6,opacity=.4](A,O,B)% square angle here
		\tkzLabelAngle[pos = 0.35](A,O,B){$\hat{a}$}
		
		\tkzMarkAngle[fill= orange,size=0.9cm,%
		opacity=.4](B,A,O)
		\tkzLabelAngle[pos = 0.7](B,A,O){$\hat{b}$}
		
		\tkzMarkAngle[fill= orange,size=0.7cm,%
		opacity=.4](O,B,A)
		\tkzLabelAngle[pos = 0.5](O,B,A){$\hat{c}$}
		\end{tikzpicture}
\end{minipage}
\begin{minipage}{0.5\linewidth}
				
		\begin{tikzpicture}[thick]
		\coordinate (O) at (-1.3,0);
		\coordinate (A) at (0.7,0);
		\coordinate (B) at (2.4,0);
		\coordinate (C) at (2.4,2.3);
		\draw (O)--(A)--(C)--cycle;
		\draw (O)--(B)--(C)--cycle;
		
		\coordinate (D) at (3.2,0);
		\coordinate (E) at (3.2,2.3);
		\draw[pattern=north west lines, pattern color=orange,  opacity=.6] (B)--(C)--(E)--(D)--cycle;
		
		\node at (O) [below left=2pt]{$A$};
		\node at (A) [below=2pt]{$B$};
		\node at (B) [below=2pt]{$C$};
		\node at (C) [above =2pt]{$D$};
		
		\tkzLabelSegment[below=2pt](O,A){$50m$}
		\tkzLabelSegment[center right=4pt, rotate=-90](E,D){Edificio}
		\tkzLabelSegment[above right=2pt](A,B){}
		
		\tkzMarkAngle[fill=orange,size=1,opacity=.4](A,O,C)
		\tkzLabelAngle[pos = 0.76](A,O,C){$30\degree$}
		
		\tkzMarkAngle[fill= orange,size=0.85,%
		opacity=.4](A,C,B)
		\tkzLabelAngle[pos = 0.65](A,C,B){$65\degree$}	
		
		\end{tikzpicture}
	\end{minipage}
	
	
	\item (bonus 1)\textbf{Extra:}
	
	\begin{minipage}{0.3\linewidth}
		
		\begin{tikzpicture}[thick]
		\coordinate (O) at (0,0);
		\coordinate (A) at (4,0);
		\coordinate (B) at (3,3);
		\coordinate (M) at (3,0);
		\draw (O)--(A)--(B)--cycle;
		\draw[dashed,  opacity=0.5] (M)--(B);
		
		\node at (O) [below left=2pt]{$A$};
		\node at (A) [right=2pt]{$B$};
		\node at (B) [above left=2pt]{$C$};
		\node at (M) [below=2pt]{$M$};
		
		\tkzLabelSegment[below=2pt](O,A){}
		\tkzLabelSegment[left=2pt](O,B){}
		\tkzLabelSegment[above right=2pt](A,B){}
		\tkzLabelSegment[left=1pt](B,M){$h$}
		
		\tkzMarkAngle[fill=orange,size=0.8,opacity=.4](A,O,B)% square angle here
		\tkzLabelAngle[pos = 0.5](A,O,B){$\hat{a}$}
		
		\tkzMarkAngle[fill= orange,size=0.8cm,%
		opacity=.4](B,A,O)
		\tkzLabelAngle[pos = 0.5](B,A,O){$\hat{b}$}
		
		\tkzMarkAngle[fill= orange,size=0.7cm,%
		opacity=.4](O,B,A)
		\tkzLabelAngle[pos = 0.5](O,B,A){$\hat{c}$}
		\end{tikzpicture}
		
	\end{minipage}
	\begin{minipage}{0.7\linewidth}
		Sabiendo que para un triangulo, el área del mismo se expresa como: 
		
		\[ Area(ABC)=\frac{1}{2}.\overline{AB}.h \] donde $\overline{AB}$ es la base del triangulo y $h$ la altura.
		
		Obtener a partir de esta relación, que \[ Area(ABC)=\frac{1}{2}.\overline{AB}.\overline{AC}.\sin(\hat{a}). \]
		
		Observar que una relación similar también se cumple para los ángulos $\hat{b}$ y $\hat{c}$, y que partiendo de este resultado se puede deducir el teorema del seno. 
	\end{minipage}
	
	\item (bonus 2)\textbf{Extra:}
	Deducir porque en el caso del triangulo rectángulo siempre resulta que $\cos(\alpha)=\sin(\beta)$ y $\sin(\alpha)=\cos(\beta)$ 
	
\end{enumerate}

\rule[2ex]{\textwidth}{1pt}

“The avarage human has one breast and one testicle.”   -Des Machale

\newpage

\noindent
	\begin{minipage}{0.92\linewidth}
		\begin{tabular*}{\textwidth}{l @{\extracolsep{\fill}} r @{\extracolsep{6pt}} l}
			\textbf{\class} & \textbf{Profesor: \examprof}\\
			\textbf{\examnumdos}  & \textbf{}   \\
			%& Teaching Assistant & \makebox[2in]{\hrulefill}
			\textbf{Nombre: } \makebox[2in]{\hrulefill} & \textbf{\examdate} 
		\end{tabular*}\\
	\end{minipage}
	\begin{minipage}[r]{0.08\linewidth}
		\begin{flushright}
			\includegraphics[width=\linewidth]{bost.png}
		\end{flushright}
	\end{minipage}\\
	\rule[2ex]{\textwidth}{2pt}
	
	\begin{center}
		\textsl{\textbf{\underline{Justificar}}} cada respuesta. El trabajo practico se entrega \textbf{\underline{escrito en tinta}}.\\
		Si se traban con un ejercicio sigan con el siguiente.
		\textbf{Preguntas:} $\bigcirc \bigcirc \bigcirc  \bigcirc $
	\end{center}
	\begin{table}[h]
		\centering
		%\caption{My caption}
		\label{my-label}
		\begin{tabular}{|l|c|c|c|}
			\hline
			Ejercicio        & 1 & 2 & Nota \\ \hline
			Puntaje máximo   & 4 & 6 &  10  \\ \hline
			Puntaje obtenido &   &   &      \\ \hline
		\end{tabular}
	\end{table}
	
	Si se traban con algún ejercicio, pasen al siguiente y vuelvan a intentar mas tarde con el que dejaron.
	
	\begin{enumerate}
		
		\item Resolver los siguientes triángulos rectángulos. (Calcular los lados, los ángulos y sus razones trigonométricas). \label{rectangulos}\\
		
		\begin{minipage}{0.45\linewidth}
			
			\begin{tikzpicture}[thick]
			\coordinate (O) at (0,0);
			\coordinate (A) at (3.5,0);
			\coordinate (B) at (0,2.6);
			\draw (O)--(A)--(B)--cycle;
			
			\tkzLabelSegment[below=2pt](O,A){$b$}
			\tkzLabelSegment[left=2pt](O,B){$a$}
			\tkzLabelSegment[above right=2pt](A,B){$h$}
			
			\tkzMarkRightAngle[fill=orange,size=0.6,opacity=.4](A,O,B)% square angle here
			\tkzLabelAngle[pos = 0.35](A,O,B){$\hat{\gamma}$}
			
			\tkzMarkAngle[fill= orange,size=0.8cm,%
			opacity=.4](B,A,O)
			\tkzLabelAngle[pos = 0.6](B,A,O){$\hat{\alpha}$}
			
			\tkzMarkAngle[fill= orange,size=0.8cm,%
			opacity=.4](O,B,A)
			\tkzLabelAngle[pos = 0.5](O,B,A){$\hat{\beta}$}
			
			\end{tikzpicture}
			
		\end{minipage}
		\begin{minipage}{0.55\linewidth}
			\begin{enumerate}
				%\item $a=3km$,  $\quad b=4km$ 		
				\item $a=2cm$, $\quad b=1cm$. Expresar los resultados de los ángulos en el sistema sexagesimal.
				%\item $a=5cm$, $\quad \sin(\beta)=\frac{1}{\sqrt{2}}$
				\item $a=5m$, $\quad \cos(\alpha)=\frac{\sqrt{2}}{2}$. Expresar los resultados de los ángulos en Radianes.
			\end{enumerate}
		\end{minipage}
		
		\item Resolver los siguientes triángulos. El esquema del triangulo es solo para que sepan como los puntos $A,B$ y $C$ se corresponden con los ángulos. \label{obtusos}
		
		
		\begin{minipage}{0.65\linewidth}
			\begin{enumerate}
				%\item $\overline{ab}=8cm$,  $\overline{ac}=12cm$ y $\overline{bc}=5cm$ . Dibujar aproximadamente el triangulo y encontrar el valor de $x= \hat{c}$.
				%\item Calcular la altura de un edificio sabiendo que a una cierta distancia la punta del edificio se encuentra a un angulo de $25\degree$ respecto del suelo, y unos $40m$ mas lejos, la misma se observa a una inclinación $15\degree$.
				\item Dos barcos parten de un puerto con una trayectoria que forma un angulo de $50\degree$ si ($\hat{a}=50\degree$). 
				Si al cabo de media hora cada barco recorre $12km$ y $23km$ respectivamente ($\overline{AC}=12km$ y $\overline{AB}=23km$). ¿Cual es la distancia entre ellos?.
				
				Dibujar aproximadamente el triangulo y encontrar el valor de $x= \hat{b}$.
				%\item $\hat{a}=60\degree$, $\hat{b}=70\degree$, y $\overline{ac}=20m$. Dibujar aproximadamente el triangulo y resolverlo (hallar todos los valores restantes).
				\item $\overline{AB}=10cm$,  $\overline{AC}=3cm$ y $\overline{BC}=4cm$ . Dibujar aproximadamente el triangulo y encontrar el valor de $x= \hat{c}$.
			\end{enumerate}
		\end{minipage}
				\begin{minipage}{0.3\linewidth}
			
			\begin{tikzpicture}[thick]
			\coordinate (O) at (0,0);
			\coordinate (A) at (3.5,0);
			\coordinate (B) at (1,2.3);
			\draw (O)--(A)--(B)--cycle;
			
			\node at (O) [below left=2pt]{$A$};
			\node at (A) [right=2pt]{$B$};
			\node at (B) [above left=2pt]{$C$};
			
			\tkzLabelSegment[below=2pt](O,A){}
			\tkzLabelSegment[left=2pt](O,B){}
			\tkzLabelSegment[above right=2pt](A,B){}
			
			\tkzMarkAngle[fill=orange,size=0.6,opacity=.4](A,O,B)% square angle here
			\tkzLabelAngle[pos = 0.35](A,O,B){$\hat{a}$}
			
			\tkzMarkAngle[fill= orange,size=0.9cm,%
			opacity=.4](B,A,O)
			\tkzLabelAngle[pos = 0.7](B,A,O){$\hat{b}$}
			
			\tkzMarkAngle[fill= orange,size=0.7cm,%
			opacity=.4](O,B,A)
			\tkzLabelAngle[pos = 0.5](O,B,A){$\hat{c}$}
			\end{tikzpicture}
		\end{minipage}
%		\begin{minipage}{0.5\linewidth}
%			
%			\begin{tikzpicture}[thick]
%			\coordinate (O) at (-1.3,0);
%			\coordinate (A) at (0.7,0);
%			\coordinate (B) at (2.4,0);
%			\coordinate (C) at (2.4,2.3);
%			\draw (O)--(A)--(C)--cycle;
%			\draw (O)--(B)--(C)--cycle;
%			
%			\coordinate (D) at (3.2,0);
%			\coordinate (E) at (3.2,2.3);
%			\draw[pattern=north west lines, pattern color=orange,  opacity=.6] (B)--(C)--(E)--(D)--cycle;
%			
%			\node at (O) [below left=2pt]{$A$};
%			\node at (A) [below=2pt]{$B$};
%			\node at (B) [below=2pt]{$C$};
%			\node at (C) [above =2pt]{$D$};
%			
%			\tkzLabelSegment[below=2pt](O,A){$40m$}
%			\tkzLabelSegment[center right=4pt, rotate=-90](E,D){Edificio}
%			\tkzLabelSegment[above right=2pt](A,B){}
%			
%			\tkzMarkAngle[fill=orange,size=1,opacity=.4](A,O,C)
%			\tkzLabelAngle[pos = 0.76](A,O,C){$15\degree$}
%			
%			\tkzMarkAngle[fill= orange,size=0.85,%
%			opacity=.4](B,A,C)
%			\tkzLabelAngle[pos = 0.59](B,A,C){$25\degree$}	
%			
%			\end{tikzpicture}
%%		\end{minipage}
%		
		
		\item (bonus 1)\textbf{Extra:}
		
		\begin{minipage}{0.3\linewidth}
			
			\begin{tikzpicture}[thick]
			\coordinate (O) at (0,0);
			\coordinate (A) at (4,0);
			\coordinate (B) at (3,3);
			\coordinate (M) at (3,0);
			\draw (O)--(A)--(B)--cycle;
			\draw[dashed,  opacity=0.5] (M)--(B);
			
			\node at (O) [below left=2pt]{$A$};
			\node at (A) [right=2pt]{$B$};
			\node at (B) [above left=2pt]{$C$};
			\node at (M) [below=2pt]{$M$};
			
			\tkzLabelSegment[below=2pt](O,A){}
			\tkzLabelSegment[left=2pt](O,B){}
			\tkzLabelSegment[above right=2pt](A,B){}
			\tkzLabelSegment[left=1pt](B,M){$h$}
			
			\tkzMarkAngle[fill=orange,size=0.8,opacity=.4](A,O,B)% square angle here
			\tkzLabelAngle[pos = 0.5](A,O,B){$\hat{a}$}
			
			\tkzMarkAngle[fill= orange,size=0.8cm,%
			opacity=.4](B,A,O)
			\tkzLabelAngle[pos = 0.5](B,A,O){$\hat{b}$}
			
			\tkzMarkAngle[fill= orange,size=0.7cm,%
			opacity=.4](O,B,A)
			\tkzLabelAngle[pos = 0.5](O,B,A){$\hat{c}$}
			\end{tikzpicture}
			
		\end{minipage}
		\begin{minipage}{0.7\linewidth}
			Sabiendo que para un triangulo, el área del mismo se expresa como: 
			
			\[ Area(ABC)=\frac{1}{2}.\overline{AB}.h \] donde $\overline{AB}$ es la base del triangulo y $h$ la altura.
			
			Obtener a partir de esta relación, que \[ Area(ABC)=\frac{1}{2}.\overline{AB}.\overline{AC}.\sin(\hat{a}). \]
			
			Observar que una relación similar también se cumple para los ángulos $\hat{b}$ y $\hat{c}$, y que partiendo de este resultado se puede deducir el teorema del seno. 
		\end{minipage}
		
		\item (bonus 2)\textbf{Extra:}
		Deducir porque en el caso del triangulo rectángulo siempre resulta que $\cos(\alpha)=\sin(\beta)$ y $\sin(\alpha)=\cos(\beta)$ 
		
	\end{enumerate}
	
	\rule[2ex]{\textwidth}{1pt}
	
	“The avarage human has one breast and one testicle.”   -Des Machale
	
	
\newpage

\noindent
	\begin{minipage}{0.92\linewidth}
		\begin{tabular*}{\textwidth}{l @{\extracolsep{\fill}} r @{\extracolsep{6pt}} l}
			\textbf{\class} & \textbf{Profesor: \examprof}\\
			\textbf{\examnumvulcano}  & \textbf{}   \\
			%& Teaching Assistant & \makebox[2in]{\hrulefill}
			\textbf{Nombre: } \makebox[2in]{\hrulefill} & \textbf{\examdate} 
		\end{tabular*}\\
	\end{minipage}
	\begin{minipage}[r]{0.08\linewidth}
		\begin{flushright}
			\includegraphics[width=\linewidth]{bost.png}
		\end{flushright}
	\end{minipage}\\
	\rule[2ex]{\textwidth}{2pt}
	
	%%%%%%%%%%%%%%%%%%%%%%%%%%%%%%%%%%%%%%%%%%%
	
	\begin{center}
		\textsl{\textbf{\underline{Justificar}}} cada respuesta. El trabajo practico se entrega \textbf{\underline{escrito en tinta}}.\\
		Si se traban con un ejercicio sigan con el siguiente.
		\textbf{Preguntas:} $\bigcirc \bigcirc \bigcirc  \bigcirc $
	\end{center}
	
	\begin{table}[h]
		\centering
		%\caption{My caption}
		\label{my-label}
		\begin{tabular}{|l|c|c|c|}
			\hline
			Ejercicio        & 1 & 2 & Nota \\ \hline
			Puntaje máximo   & 4 & 6 &  10  \\ \hline
			Puntaje obtenido &   &   &      \\ \hline
		\end{tabular}
	\end{table}
	
	Si se traban con algún ejercicio, pasen al siguiente y vuelvan a intentar mas tarde con el que dejaron.
	
	\begin{enumerate}
		
		\item Resolver los siguientes triángulos rectángulos. (Calcular los lados, los ángulos y sus razones trigonométricas). \label{rectangulos}\\
		
		\begin{minipage}{0.45\linewidth}
			
			\begin{tikzpicture}[thick]
			\coordinate (O) at (0,0);
			\coordinate (A) at (3.5,0);
			\coordinate (B) at (0,3);
			\draw (O)--(A)--(B)--cycle;
			
			\tkzLabelSegment[below=2pt](O,A){$b$}
			\tkzLabelSegment[left=2pt](O,B){$a$}
			\tkzLabelSegment[above right=2pt](A,B){$h$}
			
			\tkzMarkRightAngle[fill=orange,size=0.6,opacity=.4](A,O,B)% square angle here
			\tkzLabelAngle[pos = 0.35](A,O,B){$\hat{\gamma}$}
			
			\tkzMarkAngle[fill= orange,size=0.8cm,%
			opacity=.4](B,A,O)
			\tkzLabelAngle[pos = 0.6](B,A,O){$\hat{\alpha}$}
			
			\tkzMarkAngle[fill= orange,size=0.8cm,%
			opacity=.4](O,B,A)
			\tkzLabelAngle[pos = 0.5](O,B,A){$\hat{\beta}$}
			
			\end{tikzpicture}
			
		\end{minipage}
		\begin{minipage}{0.55\linewidth}
			\begin{enumerate}
				\item $a=3km$,  $\quad b=4km$ 		
				%\item $a=2cm$, $\quad b=1cm$
				\item $a=5cm$, $\quad \sin(\beta)=0.5$
			%\item $a=5cm$, $\quad \cos(\alpha)=\frac{\sqrt{2}}{2}$ 
		\end{enumerate}
	\end{minipage}
	
	\item Resolver los siguientes triángulos. El esquema del triangulo es solo para que sepan como los puntos $A,B$ y $C$ se corresponden con los angulos. \label{obtusos}
	%aca va una tabla y unos graficos
	
	\begin{minipage}{0.35\linewidth}
		
		\begin{tikzpicture}[thick]
		\coordinate (O) at (0,0);
		\coordinate (A) at (3.5,0);
		\coordinate (B) at (-0.6,3);
		\draw (O)--(A)--(B)--cycle;
		
		\node at (O) [below left=2pt]{$A$};
		\node at (A) [right=2pt]{$B$};
		\node at (B) [above left=2pt]{$C$};
		
		\tkzLabelSegment[below=2pt](O,A){}
		\tkzLabelSegment[left=2pt](O,B){}
		\tkzLabelSegment[above right=2pt](A,B){}
		
		\tkzMarkAngle[fill=orange,size=0.6,opacity=.4](A,O,B)% square angle here
		\tkzLabelAngle[pos = 0.35](A,O,B){$\hat{a}$}
		
		\tkzMarkAngle[fill= orange,size=0.9cm,%
		opacity=.4](B,A,O)
		\tkzLabelAngle[pos = 0.7](B,A,O){$\hat{b}$}
		
		\tkzMarkAngle[fill= orange,size=0.7cm,%
		opacity=.4](O,B,A)
		\tkzLabelAngle[pos = 0.5](O,B,A){$\hat{c}$}
		\end{tikzpicture}
		
	\end{minipage}
	\begin{minipage}{0.65\linewidth}
		\begin{enumerate}
			\item $\overline{AB}=8cm$,  $\overline{AC}=12cm$ y $\overline{BC}=8cm$ . Dibujar aproximadamente el triangulo y encontrar el valor de $x= \hat{c}$.
			\item $\hat{c}=40\degree$,  $\overline{AC}=30cm$ y $\overline{AB}=20cm$ . Dibujar aproximadamente el triangulo y encontrar el valor de $x= \hat{b}$.
			%\item $\hat{a}=60\degree$, $\hat{b}=70\degree$, y $\overline{ac}=20m$. Dibujar aproximadamente el triangulo y resolverlo (hallar todos los valores restantes).
			%\item $\overline{ab}=10cm$,  $\overline{ac}=7cm$ y $\overline{bc}=8cm$ . Dibujar aproximadamente el triangulo y encontrar el valor de $x= \hat{c}$.
		\end{enumerate}
	\end{minipage}
	
	\item (bonus 1)\textbf{Extra:}
	
	\begin{minipage}{0.35\linewidth}
		
		\begin{tikzpicture}[thick]
		\coordinate (O) at (0,0);
		\coordinate (A) at (4,0);
		\coordinate (B) at (3,3);
		\coordinate (M) at (3,0);
		\draw (O)--(A)--(B)--cycle;
		\draw[dashed,  opacity=0.5] (M)--(B);
		
		\node at (O) [below left=2pt]{$A$};
		\node at (A) [right=2pt]{$B$};
		\node at (B) [above left=2pt]{$C$};
		\node at (M) [below=2pt]{$M$};
		
		\tkzLabelSegment[below=2pt](O,A){}
		\tkzLabelSegment[left=2pt](O,B){}
		\tkzLabelSegment[above right=2pt](A,B){}
		\tkzLabelSegment[left=1pt](B,M){$h$}
		
		\tkzMarkAngle[fill=orange,size=0.8,opacity=.4](A,O,B)% square angle here
		\tkzLabelAngle[pos = 0.5](A,O,B){$\hat{a}$}
		
		\tkzMarkAngle[fill= orange,size=0.8cm,%
		opacity=.4](B,A,O)
		\tkzLabelAngle[pos = 0.5](B,A,O){$\hat{b}$}
		
		\tkzMarkAngle[fill= orange,size=0.7cm,%
		opacity=.4](O,B,A)
		\tkzLabelAngle[pos = 0.5](O,B,A){$\hat{c}$}
		\end{tikzpicture}
		
	\end{minipage}
	\begin{minipage}{0.65\linewidth}
		Sabiendo que para un triangulo, el área del mismo se expresa como: 
		
		\[ Area(ABC)=\frac{1}{2}.\overline{AB}.h \] donde $\overline{AB}$ es la base del triangulo y $h$ la altura.
		
		Obtener a partir de esta relación, que \[ Area(ABC)=\frac{1}{2}.\overline{AB}.\overline{AC}.\sin(\hat{a}). \]
		
		Observar que una relación similar también se cumple para los ángulos $\hat{b}$ y $\hat{c}$, y que partiendo de este resultado se puede deducir el teorema del seno. 
	\end{minipage}
	
	\item (bonus 2)\textbf{Extra:}
	Deducir porque en el caso del triangulo rectángulo siempre resulta que $\cos(\alpha)=\sin(\beta)$ y $\sin(\alpha)=\cos(\beta)$ 
	
\end{enumerate}

\rule[2ex]{\textwidth}{2pt}

“The avarage human has one breast and one testicle.”   -Des Machale

%\section*{Respuestas}

%1: a)3 b)49 c)-3 d)$ 1/16$ e)2 f)$ 1+0,68 $ g)$ 1,46$ h)$2.0,68$ 

%2: a)27 b)2  c)2 d)6 %d)$\frac{5}{3.c^2}$

%3: 1.
%\begin{figure}[h!]
%\centering
%\includegraphics[width=0.7\textwidth]{encontrarlog2xmenos2.jpg}
%\caption{}
%\label{fig:logaritmo}
%\end{figure}

%3: 2.$log_2(x-2)$



\end{document}
