\documentclass[a4paper,spanish]{exam}
\usepackage[spanish]{babel}
%\usepackage[utf8]{inputenc}
\usepackage{multicol}
%\usepackage[latin1]{inputenc}
\usepackage{fontspec}%la posta para las tildes con lualatex
\usepackage[margin=0.5in]{geometry}
\usepackage{amsmath,amssymb}
\usepackage{multicol}
\usepackage{natbib}
\usepackage{graphicx}
\usepackage{hyperref}
\usepackage{epstopdf}
\usepackage{capt-of}
\usepackage[usenames]{color}
%los de aca abajo capaz no los uso
\newcommand{\class}{Matemática: Evaluación de  Logaritmos}
\newcommand{\term}{2° Trimestre 2015}
\newcommand{\examnum}{Tema 3}
\newcommand{\examprof}{Alexis Gomel}
\newcommand{\examdate}{12/8/2015}
\newcommand{\timelimit}{60 Minutes}%no lo uso

%el header de las hojas.
\pagestyle{head}
\firstpageheader{}{}{}
\runningheader{\class}{\examnum\ - Page \thepage\ of \numpages}{\examdate}
\runningheadrule


\begin{document}
\noindent
\begin{tabular*}{\textwidth}{l @{\extracolsep{\fill}} r @{\extracolsep{6pt}} l}
\textbf{\class} & \textbf{Profesor: \examprof}\\
\textbf{\examnum} & \textbf{\examdate} \\
%\textbf{Time Limit: \timelimit} & Teaching Assistant & \makebox[2in]{\hrulefill}
\textbf{Nombre: } \makebox[2in]{\hrulefill}
\end{tabular*}\\
\rule[2ex]{\textwidth}{2pt}

%%%%%%%%%%%%%%%%%%%%%%%%%%%%%%%%%%%%%%%%%%%

Justificar cada respuesta. El examen esta pensado para que no haga falta usar una calculadora.

\begin{table}[h]
\centering
%\caption{My caption}
\label{my-label}
\begin{tabular}{|l|c|c|c|c|}
\hline
Ejercicio        & 1 & 2 & 3 & Nota \\ \hline
Puntaje máximo   & 4 & 4 & 2 &   10   \\ \hline
Puntaje obtenido &   &   &   &      \\ \hline
\end{tabular}
\end{table}

Si se traban con algún ejercicio, pasen al siguiente y vuelvan a intentar mas tarde con el que dejaron.


\begin{enumerate}
\item (4 Puntos)\textbf{Resolver:} 
\begin{multicols}{2}
\begin{enumerate}
\item $log(100)-log_{\frac{1}{2}}(1)$
\item $3^2.{log_3(7)}$
\item $log_2(\frac{1}{32})$

\columnbreak

Sabiendo que $log_2(5)\simeq 2,32$, calcular:

\item $log_2(10)$
\item $log_5(2)$
\item $log_2(25)$


\end{enumerate}
\end{multicols}



\item (4 Puntos)\textbf{Encontrar, si es posible, el valor de x :}
\begin{enumerate}
\item $log(x)=2.log(4)$
\item $log_5(3.x-1)=1$
\item $12 \cdot 4^{x}- 9 \cdot 4^x=48$
%\item $log(x)=log(\frac{5.c}{3})-3.log(c)$
\end{enumerate}

\item (2 Puntos)\textbf{Gráficos:}
Cada ítem vale 1 punto.
\begin{enumerate}
%\begin{multicols}{2}

\item  Graficar $y=log_{2}(x-1)$. (Basta con completar la tabla, y unir los puntos.)

Indicar en que  valor de $x$ esta la asíntota vertical. 

%\columnbreak

%\begin{table}[h]
%\centering
%\caption{Tabla para Graficar}
%%\label{my-label}
%\begin{tabular}{|c|c|}
%\hline
%$x$ & $y$ \\ \hline
%3 &     \\ \hline
%4  &    \\ \hline
%6  &    \\ \hline
%10    &    \\ \hline
%$\frac{5}{2} $ &    \\ \hline
%$\frac{9}{4} $ &    \\ \hline
%$\frac{17}{8} $ &    \\ \hline
%\end{tabular}
%\end{table}

\begin{table}[h]
\centering
%\caption{Tabla para Graficar}
\label{my-label}
\begin{tabular}{|c|c|c|c|c|c|c|}
\hline
$x$ & 2 & 3 & 5 & 9 & $3/2$ & $7/4 $  \\ \hline
$y$ & & & & & &  \\ \hline
\end{tabular}
\end{table}

%\end{multicols}

\item  Encontrar $a$ y $b$ ,  a partir del gráfico de $y=log_a(x-b)$.



\begin{figure}[h!]
\centering
\includegraphics[width=0.7\textwidth]{encontrarlog3xmas2.jpg}
\caption{Encontrar $a$ y $b$,  a partir del gráfico de $y=log_a(x-b)$.
Los puntos marcados con asterisco, son los valores de $y$ cuando $x$ vale $-2,000001  ;  -1;  0;  1;  2;  3 ... $}
\label{fig:logaritmo}
\end{figure}

Pista: Analizar que pasa en $(-1,0)$ y en $(1,1)$. Que tienen que cumplir $a$ y $b$ para que sea posible que la función tome estos valores?

\end{enumerate}


 \item (bonus)\textbf{Extra:}
 Si ya terminaste los demás, este ejercicio sirve como un bonus para darte un empujón si estas cerca de aprobar, o para redondear la nota para arriba.\\
 
 Sabiendo que, por definición, $x=a^{log_a(x)}$; y $x=c^{log_c(x)}$. Demostrar que $log_a(x)=\frac{log_c(x)}{log_c(a)}$.
 
 \end{enumerate}
 
 \rule[2ex]{\textwidth}{2pt}
 
"There’s as many atoms in a single molecule of your DNA as there are stars in the typical galaxy. We are, each of us, a little universe."
$―$ Neil deGrasse Tyson, Cosmos 

%\newpage

%\section*{Respuestas}

%1: a)3 b)49 c)-3 d)$ 1/16$ e)2 f)$ 1+0,68 $ g)$ 1,46$ h)$2.0,68$ 

%2: a)27 b)2  c)2 d)6 %d)$\frac{5}{3.c^2}$

%3: 1.
%\begin{figure}[h!]
%\centering
%\includegraphics[width=0.7\textwidth]{encontrarlog2xmenos2.jpg}
%\caption{}
%\label{fig:logaritmo}
%\end{figure}

%3: 2.$log_2(x-2)$



\end{document}
